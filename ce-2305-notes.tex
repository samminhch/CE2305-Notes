\documentclass{MathNotes}

\newenvironment{example}[1]{\begin{BlueBox}{Example}{#1}}{\end{BlueBox}}
\newenvironment{definition}[1]{\begin{RedBox}{Definition}{#1}}{\end{RedBox}}
\newenvironment{note}[1]{\begin{YellowBox}{Note}{#1}}{\end{YellowBox}}
\newenvironment{theorem}[1]{\begin{GrayBox}{Theorem}{#1}}{\end{GrayBox}}
\newenvironment{practice}[1]{\begin{PurpleBox}{Practice}{#1}}{\end{PurpleBox}}
    \newenvironment{graph}{\begin{tikzpicture}[node distance={0.65in}, thick, auto=center,every node/.style={draw,circle,fill=green!10},other/.style={draw,circle,fill=blue!10},blank/.style={draw=white!0,fill=green!0}]}{\end{tikzpicture}}

\newcommand{\br}{
	\begin{center}
		\line(1,0){4ex}
	\end{center}
}
\newcommand{\bl}{
	\newline$ $\newline
}
\newcommand{\continued}{
	\mbox{}
	\vfill
	\textbf{Continued on Next Page\ldots}\newpage
}
%----------------------Author Information-----------------
\title{Speed Running Discrete Mathematics}
\author{Minh Nguyen}

%---------------------Document Begin-----------------------
% \usepackage{layout}
\begin{document}
% \begin{center}
%     \layout*
% \end{center}
\newpage
\maketitle
\pagenumbering{gobble}
\vfill
\tableofcontents
\newpage
\section*{Author's Notes}
This actually isn't the first copy of my discrete math notes\ldots I had an earlier copy that got corrupted, so now I'm just going to constantly upload to GitHub because I'm afraid of them getting wiped again \emoji{sob}
\br{}

These notes are based off of the textbook \textit{Discrete Mathematics and Its Applications} by \textit{Kenneth H. Rosen}, as well as from the lectures of \textit{Serdar Erbatur} from the \textit{Fall 2023} semester.
\br{}
If you have any complaints, or suggestions regarding these notes, please
email me at \newline\href{mailto:minh.nguyen7@utdallas.edu}{mdn220004@utdallas.edu}
\newpage
\pagenumbering{arabic}

\section{Propositional Logic}\label{sec:propositional-logic}
\begin{definition}{Proposition}
	Declarative statements that are either \texttt{true} or \texttt{false}
\end{definition}

\begin{table}[h!]\label{tab:prop-examples}
	\centering
	\caption{Examples of propositions}
	\begin{tabular}{lc}
		\multicolumn{1}{c}{\textbf{Statement}} & \multicolumn{1}{c}{\textbf{Proposition?}} \\
		\midrule
		$1+1=2$                                & \emoji{check-mark}                        \\
		What class are ya takin?               & \emoji{no-entry}                          \\
		I am happy                             & \emoji{check-mark}
	\end{tabular}
\end{table}

Propositions can be combined to form another proposition with the use of \textit{logical operators}
\begin{note}{}
	Common labels for propositions in discrete mathematics include letters like $P,Q,R,S$\ldots
\end{note}

\begin{table}[h!]\label{tab:logical-operators}
	\centering
	\caption{Logical Operators in Discrete Mathematics}
	\begin{tabular}{clc}
		\multicolumn{1}{c}{\textbf{Symbol}}   &
		\multicolumn{1}{c}{\textbf{Meaning} } &
		\multicolumn{1}{c}{\textbf{Expression}}                                            \\
		\midrule
		$\lnot$                               & negation / not             & $\lnot P$     \\
		$\land$                               & conjunction / and          & $P\land Q$    \\
		$\lor$                                & disjunction / or           & $P\lor Q$     \\
		$\oplus$                              & exclusive disjunction/ xor & $P\oplus Q$   \\
		$\implies$                            & implication / conditional  & $P\implies Q$ \\
		$\iff$                                & biconditional              & $P\iff Q$     \\
	\end{tabular}
\end{table}

\begin{note}{Logical Operator Precedence}\label{order-of-operations}
	Order matters when it comes to evaluating logical operators:
	\begin{enumerate}
		\item negation
		\item conjunction
		\item disjunctions
		\item conditionals
	\end{enumerate}
	Where negation is evaluated first, and conditionals last.
\end{note}

\newpage
\begin{multicols}{2}
	\begin{theorem}{More on Conditionals}\label{th:conditionals-extra}
		Conditionals can also be expressed in three other ways:
		\bl
		\textbf{Contrapositive:} $\lnot Q\implies\lnot P$
		\bl
		\textbf{Converse:} $Q\implies P$
		\bl
		\textbf{Inverse:} $\lnot P\implies\lnot Q$
	\end{theorem}
	\begin{example}{}
		\textbf{Conditional:} If \underline{it is raining}, then \underline{I am not going to town}
		\bl
		\textbf{Contrapositive:} If \underline{I go to town}, then \underline{it is not raining}
		\bl
		\textbf{Converse:} If \underline{I do not go to town}, then \underline{it is raining}
		\bl
		\textbf{Inverse:} If \underline{it is not raining}, then \underline{I am going to town}
	\end{example}
\end{multicols}

\begin{definition}{Truth Table}
	They are used as a way of seeing all possible values of a proposition
\end{definition}

\begin{table}[h!]\label{tab:truth-table-example}
	\centering
	\caption{Truth Table of $P\implies Q$}
	\begin{tabular}{cc|c}
		$P$ & $Q$ & $P\implies Q$ \\
		\hline
		F   & F   & T             \\
		F   & T   & T             \\
		T   & F   & F             \\
		T   & T   & T             \\
	\end{tabular}
\end{table}

\subsection{Propositional Equivalencies}\label{sec:prop-equivalencies}
\begin{definition}{Equivalencies}\label{def:equivalencies}
	A proposition is a\ldots
	\begin{itemize}
		\item \textbf{Tautology} if it is \texttt{true} in \textit{every case}
		\item \textbf{Contradiction / Fallacy} if it is \texttt{false} in \textit{every case}
		\item \textbf{Contingency} if \textit{neither} is the case
	\end{itemize}
\end{definition}

\begin{table}[h!]\label{tab:tautology-contradiciton-ex}
	\centering
	\caption{Example of Tautology and Contradiction}
	\begin{tabular}{c|c|c|c}
		$P$ & $\lnot P$ & $P\lor\lnot P$ & $P\land\lnot P$ \\
		\hline
		T   & F         & T              & F               \\
		F   & T         & T              & F
	\end{tabular}
\end{table}

\begin{definition}{Logical Equivalency ($\equiv$)}\label{th:logical-equiv}
	Compound propositions $P$ and $Q$ are \textit{logically equivalent} if $P\iff Q$ is a \textit{tautology}. This is expressed as \[P\equiv Q\]
\end{definition}

Below is a table of useful logical equivalencies

\begin{table}[h!]
	\centering
	\caption{Logical Equivalencies}
	\begin{tabular}{ll}
		\textbf{Name}                 & \textbf{Equivalence}                            \\
		\midrule
		\multirow{2}{*}{Identity}     & $P\land\text{T} \equiv P$                       \\
		                              & $P\lor\text{F} \equiv P$                        \\
		\hdashline{}
		\multirow{2}{*}{Idempotent}   & $P\land P \equiv P$                             \\
		                              & $P\lor P \equiv P$                              \\
		\hdashline{}
		\multirow{2}{*}{Domination}   & $P\lor\text{T} \equiv \text{T}$                 \\
		                              & $P\land\text{F} \equiv \text{F}$                \\
		\hdashline{}
		\multirow{2}{*}{Negation}     & $P\lor\lnot P \equiv \text{T}$                  \\
		                              & $P\land\lnot P \equiv \text{F}$                 \\
		\hdashline{}
		Double Negation               & $\lnot(\lnot P)\equiv P$                        \\
		\hdashline{}
		\multirow{2}{*}{Commutative}  & $P\land Q\equiv Q\land P$                       \\
		                              & $P\lor Q\equiv Q\lor P$                         \\
		\hdashline{}
		\multirow{2}{*}{Associative}  & $(P\land Q)\land R\equiv P\land(Q\land R)$      \\
		                              & $(P\lor Q)\lor R\equiv P\lor(Q\lor R)$          \\
		\hdashline{}
		\multirow{2}{*}{Distributive} & $P\land(Q\lor R)\equiv(P\land Q)\lor(P\land R)$ \\
		                              & $P\lor(Q\land R)\equiv(P\lor Q)\land(P\lor R)$  \\
		\hdashline{}
		\multirow{2}{*}{De Morgan's}  & $\lnot(P\land Q)\equiv\lnot P\lor\lnot Q$       \\
		                              & $\lnot(P\lor Q)\equiv\lnot P\land\lnot Q$       \\
		\hdashline{}
		\multirow{2}{*}{Absorption}   & $P\land(P\lor Q)\equiv P$                       \\
		                              & $P\lor(P\land Q)\equiv P$                       \\
	\end{tabular}
\end{table}

The use of showing the equivalencies between two compound propositions is called a
\textit{conditional-disjunction equivalence}
\begin{example}{Conditional-Disjunction Equivalence}\label{ex:conditional-disjunction-equivalence}
	\[P\implies Q\equiv\lnot P\lor Q\]
\end{example}

This can be proven with the use of a truth table \emoji{smiling-face-with-hearts}
\continued

\begin{table}[h!]\label{tab:conditional-disjunction-equivalence}
	\centering
	\caption{Conditional-Disjunction Equivalence Proof}
	\begin{tabular}{cc|c|c|c}
		$P$ & $Q$ & $\lnot P$ & $\lnot P\lor Q$ & $P\implies Q$ \\
		\midrule
		F   & F   & T         & T               & T             \\
		F   & T   & T         & T               & T             \\
		T   & F   & F         & F               & F             \\
		T   & T   & F         & T               & T             \\
	\end{tabular}
\end{table}

\subsection{Predicates and Quantifiers}\label{sec:predicates-quantifiers}

\begin{note}{}
	So far, propositional logic can only handle \textit{singular} subjects. It can't handle statements such as:
	\begin{itemize}
		\item All computer science students can program well
		\item $3x+4\geq0$
	\end{itemize}
\end{note}

\begin{multicols}{2}
	\begin{definition}{Predicate Logic}\label{def:pred-logic}
		When you have a proposition that contains a variable. It is typically written as a \textit{propositional function}, $P(x)$, where $x$ is the subject for the predicate $P$
	\end{definition}

	\begin{example}{}\label{ex:pred-logic}
		\begin{itemize}
			\item $F(x)=\text{``}x>3\text{''}$
			\item $P(x)=\text{``}x\text{ looks beautiful!''}$
		\end{itemize}
	\end{example}
\end{multicols}

\begin{note}{Multi-variable Propositional Functions}\label{note:prop-func-multi}
	Propositional functions may also contain more than just one argument:
	\[P(x,y)\]
\end{note}

\begin{definition}{Quantifiers}\label{def:quantifiers}
	They define the range of which a proposition holds \texttt{true}, and can be nested to produce \textit{nested quantifiers}
\end{definition}

\marginnote{$\forall x.P(x)$ may also be written as $\forall xP(x)$, and the same holds for the other quantifiers}
\begin{table}[h!]\label{tab:quantifiers}
	\centering
	\caption{Quantifiers in Discrete Mathematics}
	\begin{tabular}{l l l}
		\multicolumn{1}{c}{\textbf{Quantifier}} &
		\multicolumn{1}{c}{\textbf{Expression}} &
		\multicolumn{1}{c}{\textbf{In English}}                                                                                         \\
		\midrule
		Universal Quantifier                    & $\forall x. P(x)$  & $P(x)$ is \texttt{true} for every $x$ in its domain              \\
		\hdashline{}
		\multirow{3}{*}{Existential Quantifier} & $\exists x. P(x)$  & There exists $x$ where $P(x)$ is \texttt{true}                   \\
		                                        & $\nexists x. P(x)$ & There does not exist $x$ where $P(x)$ is \texttt{true}           \\
		                                        & $\exists! x. P(x)$ & There exists \textit{only one} $x$ where $P(x)$ is \texttt{true}
	\end{tabular}
\end{table}

\begin{example}{Nested Quantifiers}\label{ex:nested-quantifiers}
	\centering
	\[\forall x\exists y.(x+y=0)\] Translates to: ``For \textit{every} $x$ \textit{there exists} a $y$ such that $x+y=0$''
\end{example}
\subsection{Inference Rules and Proofs}
\begin{multicols}{2}
	\begin{definition}{Argument}\label{def:argument}
		A sequence of statements that have a conclusion
	\end{definition}

	\begin{definition}{Valid}\label{def:valid}
		The conclusion, the final statement of the argument, must follow from its premises
		\newline
		(i.e. premises $\implies$ conclusion)
	\end{definition}

	\begin{definition}{Premise}\label{def:premise}
		The preceding statements of a mathematical argument that lead to a conclusion
	\end{definition}

	\begin{definition}{Fallacy}
		Incorrect reasoning in discrete mathematics that leads to an invalid argument
	\end{definition}
\end{multicols}


\begin{example}{Argumentative Form}\label{ex:argumentative-form}
	Arguments may be written as this: $\big((P\implies Q)\land P\big)\implies Q$, or in argumentative form:
	\begin{align*}
		           & P\implies Q                 \\
		           & \underline{P\hspace{3.5em}} \\
		\therefore & Q
	\end{align*}
\end{example}

The next page contains a table of inference rules
\newpage
\begin{table}[h!]\label{tab:inference-rules}
	\centering
	\caption{Rules of Inference}
	\begin{tabular}{lll}
		\multicolumn{1}{c}{\textbf{Rule}}                          &
		\multicolumn{1}{c}{\textbf{Expression}}                    &
		\multicolumn{1}{c}{\textbf{Tautology}}                            \\
		\midrule
		Modus ponens                                               &
		$\begin{array}{rl}
				            & P                       \\
				            & \underline{P\implies Q} \\
				 \therefore & Q\end{array}$                    &
		$\big(P\land(P\implies Q)\big)\implies Q$                         \\
		\hdashline{}
		Modus tollens                                              &
		$\begin{array}{rl}
				            & \lnot Q                 \\
				            & \underline{P\implies Q} \\
				 \therefore & \lnot P\end{array}$                    &
		$\big(\lnot Q\land(P\implies Q)\big)\implies \lnot P$             \\
		\hdashline{}
		Hypothetical syllogism                                     &
		$\begin{array}{rl}
				            & P\implies Q             \\
				            & \underline{Q\implies R} \\
				 \therefore & P\implies R\end{array}$                    &
		$\big((P\implies Q)\land(Q\implies R)\big)\implies (P\implies R)$ \\
		\hdashline{}
		Disjunctive syllogism                                      &
		$\begin{array}{rl}
				            & P\lor Q                            \\
				            & \underline{\lnot P\hspace{1.25em}} \\
				 \therefore & Q\end{array}$ &
		$\big((P\lor Q)\land\lnot P\big)\implies Q$                       \\
		\hdashline{}
		Addition                                                   &
		$\begin{array}{rl}
				            & \underline{P\hspace{2em}} \\
				 \therefore & P\lor Q\end{array}$       &
		$P\implies(P\lor Q)$                                              \\
		\hdashline{}
		Simplification                                             &
		$\begin{array}{rl}
				            & \underline{P\land Q} \\
				 \therefore & P\end{array}$                       &
		$(P\land Q)\implies P$                                            \\
		\hdashline{}
		Conjunction                                                &
		$\begin{array}{rl}
				            & P                            \\
				            & \underline{Q\hspace{1.25em}} \\
				 \therefore & P\land Q\end{array}$       &
		$\big((P)\land(Q)\big)\implies P\land Q$                          \\
		\hdashline{}
		Resolution                                                 &
		$\begin{array}{rl}
				            & P\lor Q                   \\
				            & \underline{\lnot P\lor R} \\
				 \therefore & Q\lor R\end{array}$                  &
		$\big((P\lor Q)\land(\lnot P\lor R)\big)\implies (Q\lor R)$       \\
		\hdashline{}
	\end{tabular}
\end{table}

\begin{table}[h!]\label{tab:inference-rules-quantified}
	\centering
	\caption{Rules of Inference for Quantified Statements}
	\begin{tabular}{ll}
		\multicolumn{1}{c}{\textbf{Rule}} &
		\multicolumn{1}{c}{\textbf{Expression}}                         \\
		\midrule
		Universal Instantiation           &
		$\begin{array}{rl}
				            & \underline{\forall x.P(x)} \\
				 \therefore & P(c)\end{array}$                        \\
		\hdashline{}
		Universal Generalization          &
		$\begin{array}{rl}
				            & \underline{P(c) \text{ for an arbitrary } c} \\
				 \therefore & \forall x.P(x)\end{array}$        \\
		\hdashline{}
		Existential Instantiation         &
		$\begin{array}{rl}
				            & \underline{\exists x.P(x)\hspace{7em}} \\
				 \therefore & P(c)\text{ for some element } c\end{array}$ \\
		\hdashline{}
		Existential Generalization        &
		$\begin{array}{rl}
				            & \underline{P(c) \text{ for an arbitrary } c} \\
				 \therefore & \forall x.P(x)\end{array}$        \\
	\end{tabular}
\end{table}

\begin{definition}{Proof}\label{def:proof}
	\underline{Valid arguments} that establish the \texttt{truth} of mathematical statements
\end{definition}

\begin{definition}{Theorem}\label{def:theorem}
	A statement or claim that can be proven using:
	\begin{itemize}
		\item definitions
		\item other theorems
		\item axioms
		\item inference rules
	\end{itemize}
	They are also referred to as ``Lemma'', ``Proposition'', or ``Corollary''
\end{definition}
There are many different ways of proving a theorem. Let's assume a conditional statement $P\implies Q$\ldots
\begin{multicols}{2}
	\begin{definition}{Direct Proof}\label{def:direct-proof}
		When the first step is the assumption that $P$ is
		\texttt{true} and the following steps that lead up to $Q$ is also \texttt{true}
	\end{definition}

	\begin{definition}{Indirect Proof}\label{def:indirect-proof}
		Proofs that do not start with the premises and end with conclusion (the opposite of a direct proof). Ways of doing direct proofs involve \textit{proof by contraposition}, \textit{proof by contradiction}, and much more\ldots
	\end{definition}

	\begin{example}{Proof by Contraposition}\label{ex:contraposition-proof}
		$P\implies Q$ can be proved \texttt{true} if $\lnot Q\implies P$, its contraposition, can also be proved. This is because a contraposition and a conditional proposition are tautologies
	\end{example}

	\begin{example}{Proof by Contradiciton}\label{ex:contradiciton-proof}
		To prove $P$, you must assume $\lnot P$ and derive that $\lnot P$ is \texttt{false}. If $\lnot P$ is \texttt{false}, then that must mean that $\lnot\lnot P$, or $P$, must be \texttt{true}
	\end{example}
\end{multicols}

% TODO: write about proof by equivalence, existence, and exhaustion
\newpage
\section{Set Theory}
\begin{multicols}{2}
	\begin{definition}{set}\label{def:set}
		A collection of \textit{unique} objects or elements
		\begin{itemize}
			\item $a\in A$ to state that $a$ is \textit{contained} in the set $A$
			\item $a\not\in A$ to indicate that $a$ \textit{is not} contained in $A$
		\end{itemize}
	\end{definition}

	\begin{note}{Defining Sets}\label{note:defining-sets}
		Roster method
		\begin{align*}
			A & = \{a,b,c,d\}                            \\
			B & = \{4,b,c,a\}                            \\
			C & = \bigl\{1, 2, \{3.0, 3.5\}, 4, 5\bigr\} \\
			D & = \{1, 2, C\}
		\end{align*}
		Set Builder Notation
		\begin{align*}
			E & = \{x\in\mathbb{N} \big| x=2k\text{ for some }k\in\mathbb{N}\} \\
			F & = \{\alpha \big| P(\alpha) \text{ is }\texttt{true}\}
		\end{align*}
	\end{note}
\end{multicols}

\begin{table}[h!]\label{tab:important-sets}
	\centering
	\caption{Important Sets}
	\begin{tabular}{lll}
		\multicolumn{1}{c}{\textbf{Set}}                           &
		\multicolumn{1}{c}{\textbf{Expansion}}                     &
		\multicolumn{1}{c}{\textbf{Description}}                         \\
		\midrule
		$\mathbb{N}$                                               &
		$\{0, 1, 2, 3,\cdots\}$                                    &
		Natural numbers                                                  \\
		$\mathbb{Z}$                                               &
		$\{\cdots,-1,0, 1,\cdots\}$                                &
		Integers                                                         \\
		$\mathbb{Z}^+$                                             &
		$\{1, 2, 3,\cdots\}$                                       &
		Positive integers                                                \\
		$\mathbb{Q}$                                               &
		$\{\frac{P}{Q}\big|P\in\mathbb{Z},Q\in\mathbb{Z},Q\neq0\}$ &
		Rational numbers                                                 \\
		$\mathbb{R}$                                               &   &
		Real numbers                                                     \\
		$\mathbb{R}^+$                                             &
		$\{x\big|x>0\}$                                            &
		Positive real numbers                                            \\
		$\mathbb{C}$                                               &   &
		Complex numbers                                                  \\
		$\varnothing$                                              &   &
		Empty set                                                        \\
	\end{tabular}
\end{table}

\newpage
\subsection{Subsets}
\begin{multicols}{2}
	\begin{theorem}{Set Equality}\label{th:set-equality}
		Given two sets $A$ and $B$:\[A\equiv B\iff\forall x.(x\in A\iff x\in B)\] is \texttt{true}
	\end{theorem}

	\begin{definition}{Subset ($\subseteq$)}
		$A$ is a subset of $B$ if \textit{all elements of} $A$ are also contained in $B$:
		\[A\subseteq B\iff\forall x.(x\in A\implies x\in B)\]
	\end{definition}
\end{multicols}

\begin{example}{Subsets}\label{ex:subsets}
	Let $A=\{2,3\},B=\{a,b,c,2,3\}$\newline
	\begin{multicols}{2}
		\begin{itemize}
			\item $A\subseteq B$  \texttt{true}\\
			\item $A\subseteq A$  \texttt{true}\\
			\item $\varnothing\subseteq A$  \texttt{true}\\
			\item $B\subseteq A$  \texttt{false}\\
		\end{itemize}
	\end{multicols}
\end{example}

\marginnote{This notation is seen in Section~\ref{def:cardinality}}
\begin{multicols}{2}
	\begin{definition}{Power Set}\label{def:power-set}
		The set of all subsets of the set\[\bigl|A\bigr|=n,\bigl|P(A)\bigr|=2^n\]
	\end{definition}

	\begin{example}{}\label{ex:power-set}
		Let $S=\{1,2\}$:\[P(S)=\bigl\{\varnothing,\{1\},\{2\},\{1,2\}\bigr\}\]
	\end{example}
\end{multicols}

\begin{definition}{Proper Subset ($\subset$)}
	It's a subset where $A\neq B$: \[A\subset B\iff\forall x.(x\in A\implies x\in B)\land A\neq B\]
\end{definition}

\subsection{Cardinality}\label{sec:cardinality}

\begin{multicols}{2}
	\begin{definition}{Cardinality}\label{def:cardinality}
		It's the number of unique elements in a set. It's denoted by $\bigl|A\bigr|$
	\end{definition}
	\begin{example}{}\label{ex:cardinality}
		\begin{tabular}{ll}
			$\big\{\{1, 2\}, 3\big\}$ & 2 elements \\
			$\{1,2,3\}=\{1,2,3,3\}$   & 3 elements \\
			$\{\varnothing\}$         & 1 element  \\
			$\{\}$                    & 0 elements \\
			$\{\varnothing\}$         & 1 element
		\end{tabular}
	\end{example}
	\begin{definition}{Intervals}\label{def:intervals}
		Sets of numbers \textit{between} two numbers $a$ and $b$ if
		$a,b\in\mathbb{R}\land a\leq b$
		\begin{itemize}
			\item $[a, b]=\{x\big|a\leq x\leq b\}$
			\item $[a, b)=\{x\big|a\leq x < b\}$
			\item $(a, b]=\{x\big|a < x\leq b\}$
			\item $(a, b)=\{x\big|a < x < b\}$
		\end{itemize}
	\end{definition}
\end{multicols}

\newpage
\begin{multicols}{2}
	\begin{definition}{Ordered Tuples}\label{def:ordered-tuples}
		\[(a_0,a_1,a_2,\cdots,a_n)\]
		An \textit{ordered collection} of unique elements
		\[(a,b,c)\neq(b,c,a)\]
	\end{definition}
	\begin{theorem}{Cartesian Products}\label{th:cartesian-products}
		Let $A$ and $B$ be sets:\[A\times B=\{(a,b)\big|a\in A,b\in B\}\]
	\end{theorem}
\end{multicols}
\begin{example}{Cartesian Product}
	Let $A=\{1,2\},B=\{a,b,c\}$:
	\[A\times B=\{(1,a),(1,b),(1,c),(2,a),(2,b),(2,c)\}\]
\end{example}

\subsection{Set Operations}
Let $A=\{1,2,3\},B=\{2,3,4,5\}$:
\begin{table}[h!]
	\centering
	\caption{Set Operations}
	\begin{tabular}{ccl}
		\multicolumn{1}{c}{\textbf{Expression}} &
		\multicolumn{1}{c}{\textbf{Meaning}}    &
		\multicolumn{1}{c}{\textbf{Result}}                                               \\
		\midrule
		$A\cup B$                               & $A$ union $B$        & $\{1,2,3,4,5\}$  \\
		$A\cap B$                               & $A$ intersection $B$ & $\{2,3\}$        \\
		$\bar{A}$                               & complement of $A$    & $\{x\not\in A\}$ \\
	\end{tabular}
\end{table}
\begin{definition}{Membership Table}\label{def:membership-table}
	It's very similar to a truth table, where \texttt{true} means that an element exists within the set, and \texttt{false} means that the element isn't.
\end{definition}

\begin{table}[h!]
	\centering
	\caption{Membership Table Example}
	\begin{tabular}{cc|c|c}
		$A$            & $B$            & $A\cup B$      & $A\cap B$      \\
		\midrule
		\texttt{false} & \texttt{false} & \texttt{false} & \texttt{false} \\
		\texttt{false} & \texttt{true}  & \texttt{true}  & \texttt{false} \\
		\texttt{true}  & \texttt{false} & \texttt{true}  & \texttt{false} \\
		\texttt{true}  & \texttt{true}  & \texttt{true}  & \texttt{true}
	\end{tabular}
\end{table}

\newpage
\subsection{Functions}

\begin{multicols}{2}
	\begin{definition}{Function}\label{def:function}
		Function $f$ is denoted from set $A$ to set $B$:
		\[f:A\to B\]
		representing a relation that assigns each element of $A$ to \textit{exactly one} element
		of $B$
	\end{definition}

	\begin{example}{}\label{ex:functions}
		Let $A=\{1,2\},B=\{a,b\}$\[A\times B=\{(1,a),(1,b),(2,a),(2,b)\}\]
		These are functions:
		\begin{itemize}
			\item $f=\{(1,a),(2,b)\}$
			\item $f=\{(1,a),(2,a)\}$
		\end{itemize}
		This isn't:
		\begin{itemize}
			\item $f=\{(1,a),(1,b)\}$
		\end{itemize}
	\end{example}
\end{multicols}

\begin{theorem}{Equality of Functions}
	Two functions $f$ and $g$ are equal if:
	\begin{itemize}
		\item They share the same domains / co-domains
		\item They assign the same element from the domain to the same element
	\end{itemize}
	Let $f:A\to B,g:C\to D$:
	\begin{itemize}
		\item $A\neq C\lor B\neq D\implies f\neq g$
		\item $A=C\lor B=D\implies f=g$
	\end{itemize}
\end{theorem}
\begin{multicols}{2}
	{
		\begin{definition}{One-to-One Functions (Injective)}\label{def:one-to-one-functions}
			A function $f:A\to B$ where all values $A$ must correspond to \textit{one and only one} element in $B$: \[\forall a\in A,\forall b\in B.\big(f(a)\neq f(b)\big)\]
		\end{definition}

		\begin{definition}{Bijection}\label{def:bijection}
			A function that is \textit{both} one-to-one \textit{and} onto
		\end{definition}
	}
	\begin{definition}{Onto Functions (Subjective)}\label{def:onto-functions}
		A function $f:A\to B$ where all values $B$ must correspond to \textit{at least an} element in $A$: \[\forall b\in B,\exists a\in A.\big(f(a)=b\big)\]
	\end{definition}
	\begin{definition}{Inverse of Function}\label{def:inverse-function}
		\[f^{-1}(x)=y\iff f(y)=x\]
	\end{definition}
\end{multicols}

\begin{definition}{Compisite Functions}\label{def:composite-function}
	\[\text{domain}(f)=\text{range}(g)\implies f\circ g(x)=f(g(x))\]
	If the composite of the function exists, then:
	\[(f\circ g)^{-1}=g^{-1}\circ f^{-1}\]
\end{definition}

\begin{multicols}{2}
	\begin{definition}{Floor ($\lfloor x\rfloor$)}\label{def:floor-function}
		The biggest integer $n\leq x$
	\end{definition}
	\begin{definition}{Ceiling ($\lceil x\rceil$)}\label{def:floor-function}
		The smallest integer $n\geq x$
	\end{definition}
\end{multicols}
\section{Algorithms}
\begin{multicols}{2}
	\begin{definition}{Algorithm}\label{def:algorithm}
		A sequence of \textit{precise} steps used to solve computational problems
	\end{definition}
	\begin{note}{Input-Output}\label{note:input-output}
		For every input instance, a correct output \textit{must} be produced
	\end{note}
\end{multicols}

\begin{algorithm}
	\caption{Binary Search Algorithm}\label{alg:binary-search}

	\KwIn{$A\gets\{A_1, A_2, \ldots, A_n\}$ where $A_i\in\mathbb{Z}$, $k\in\mathbb{Z}$}
	\KwOut{$i$ where $A_i=k$, or $i=0\implies k\notin A$}

	left$\gets 1$\;
	right$\gets n$\;
	\While{$left<right$}
	{
		$i\gets\lfloor(\text{left}+\text{right})\div 2\rfloor$\;
		\uIf{$A_i=k$}
		{
			\KwRet~$i$\;
		}\uElseIf{$A_i<k$}
		{
			left$\gets i+1$\;
		}\Else{
			right$\gets i$\;
		}
	}
	\KwRet~$0$\;
\end{algorithm}

\begin{algorithm}
	\caption{Insertion Sort}\label{alg:insertion-sort}
	\SetKwFunction{FSwap}{swap}
	\SetKwFunction{FSort}{sort}
	\KwIn{$A\gets\{A_1, A_2, \ldots, A_n\}$ where $n\geq2$}
	\KwOut{$A\gets\{A_1, A_2, \ldots, A_n\}$ where $n\geq2\land A_i\leq A_{i+1}$}

	\For{$\text{i}\gets2$ \KwTo$n$}
	{
		key$\gets A_{\text{i}}$\;
		index$\gets\text{i}-1$\;

		\While{$\text{index}\geq1\land A_{\text{index}}>key$}
		{
			$A_{\text{index}+1}\gets A_{\text{index}}$\;
			$\text{index}\gets\text{index}-1$\;
		}
		$A_{\text{index}+1}\gets\text{key}$\;
	}

\end{algorithm}


\newpage
\subsection{Growth of Functions}
This can be used to determine the growth of functions, and is also mainly
used in the field of \textit{computer science} to analyze algorithms.

\begin{multicols}{2}
	\begin{note}{Proving Big-O}\label{note:proving-big-o}
		To prove that $f(x)=O\big(g(x)\big)$, you must prove the existence of a $C$ and $k$ that fulfill the definition:\[\exists C,k.\big(\forall x>k.(|f(x)|\leq C\times |g(x)|)\big)\]
	\end{note}
	\begin{definition}{Big-O Notation}\label{def:big-o-notation}
		Let $f$ and $g$ be functions. $f(x)$ is $O\big(g(x)\big)$ (read ``$f(x)$ is Big-O of $g(x)$'') when there are constants $C$ and $k$ (the witnesses) such that:\[|f(x)|\leq C|g(x)|,x>k\] This is called the worst-case runtime of an algorithm
	\end{definition}
\end{multicols}

\begin{example}{Identify the Witnesses}\label{ex:find-witnesses}
	Find $C,k$ such that $f(x)=4x^2+2x+1$ is $O(x^2)$
	\begin{align*}
		g(x) & = n^2+n^2+n^2 \\
		     & = 3n^2
	\end{align*}
	$f(x)<g(x) \text{ when } x>1$
	\centering $\displaystyle\therefore C=3,k=1$
\end{example}

\begin{multicols}{2}
	\begin{definition}{Big-$\Omega$}\label{def:big-omega}
		Let $f$ and $g$ be functions. $f(x)$ is $\Omega\big(g(x)\big)$ when there are
		constants $C$ and $k$ such that:\[|f(x)|\geq C|g(x)|\] This is called the
		best-case runtime of an algorithm
	\end{definition}
	\begin{definition}{Big-$\theta$}\label{def:big-theta}
		Let $f$ and $g$ be functions. $f(x)$ is $\theta\big(g(x)\big)$ when:\[f(x)=O\big(g(x)\big)\land f(x)=\Omega\big(g(x)\big)\]
		This is called the average-case runtime of an algorithm
	\end{definition}
\end{multicols}

\newpage
\section{Number Theory}\label{sec:number-theory}
\begin{definition}{Division}\label{def:division}
	Let $a,b\in\mathbb{Z}$. $a$ divides $b$ ($a|b$) if there exists integer $c$ such that $a\times c=b$
	\[a|b,a,b\in\mathbb{Z}\iff\exists c\in\mathbb{Z}.(a\times c=b)\]
\end{definition}

\begin{theorem}{Basic Division Properties}\label{th:basic-division-properties}
	Let $a,b,c\in\mathbb{Z},a\neq 0$:
	\begin{itemize}
		\item $\displaystyle (a|b)\land(a|c)\implies a|(b+c)$
		\item $\displaystyle a|b\implies \forall c\in\mathbb{Z}.\big(a|(b\times c)\big)$
		\item $\displaystyle (a|b)\land(b|c)\implies a|c$
	\end{itemize}
\end{theorem}

\begin{theorem}{Consequence of Division}\label{th:division-consequence}
	\[\forall a,b,c,m,n\in\mathbb{Z},a\neq0.\Big((a|b)\land(a|c)\implies\big(a|(mb+nc)\big)\Big)\]
	Let $a,b,c,m,n$ be integers where $a\neq0$, and suppose $a|b\land a|c$.
	Then, for any integers m and n, $a|mb+nc$.
\end{theorem}

\begin{theorem}{Division Algorithm}\label{th:division-algorithm}
	Let $a\in\mathbb{Z},d\in\mathbb{Z}^+$. Then there are unique integers $q$
	and $r$ where $0\leq r<d$ such that $a=q\times d+r$
	\[\forall a,q,r\in\mathbb{Z},d\in\mathbb{Z}^+,0\leq r<d.(a=q\times d+r)\]
\end{theorem}

\begin{multicols}{2}
	{
		\begin{definition}{Quotient}\label{def:quotient}
			\[q=a\texttt{ div }d\]
		\end{definition}
		\begin{definition}{Congruency ($\equiv$)}\label{def:congruency}
			Let $a,b\in\mathbb{Z},m\in\mathbb{Z}^+$. $a$ is congruent to $b\mod m$ if\ldots
			\begin{itemize}
				\item$m|a-b$
				\item$a\mod m\equiv b\mod m$
				\item$k\in\mathbb{Z}.(a=b+k\times m$)
			\end{itemize}
		\end{definition}
	}
	{
		\begin{definition}{Remainder}\label{def:remainder}
			\[r=a\mod d\]
		\end{definition}

		\begin{theorem}{Preservation of Congruencies}\label{th:preservation-of-congruencies}
			Let $a,b,c,d\in\mathbb{Z},m\in\mathbb{Z}^+$. If $a\equiv b\mod m\land c\equiv d\mod m$, then
			\begin{itemize}
				\item$a+c\equiv b+d\hspace{1em}(\mod m)$
				\item$a\times c\equiv b\times d\hspace{1em}(\mod m)$
			\end{itemize}
		\end{theorem}
	}
\end{multicols}
\section{Induction and Recursion}\label{sec:induction-recursion}

\begin{definition}{Proof via. Induction}\label{def:induction-proof}
	To prove that $P(n)$ is true for $n\in\mathbb{Z}^+$, we must complete these
	two steps:
	\begin{itemize}
		\item \textbf{Basis Step:} Verify that $P(1)$ is \texttt{true}
		\item \textbf{Inductive Step:} Show that
		      $P(k)\implies P(k+1),k\in\mathbb{Z}^+$ is \texttt{true}
	\end{itemize}
\end{definition}

\begin{example}{Sum to Squares}
	Prove that the sum of the first positive odd numbers is $n^2$:
	\[\displaystyle P(n)=\left(1+3+5+\cdots+2n-1=n^2\right)\]
	\br{}
	\underline{\textbf{Basis Step:}} $n=1$\newline
	\begin{align*}
		1=               & \hspace{0.5em}{(1)}^2 \\
		\therefore P(1)= & \texttt{ true}
	\end{align*}
	\underline{\textbf{Inductive Step:}} Prove $P(n)\implies P(n+1)$\newline
	Assume that $P(n)$ is true, and prove that $P(n+1)$ is also true:
	\[1+3+\cdots+2n-1+2n+1=n^2+2n+1=\left(n+1\right)^2\]
	$1+3+\cdots+2n-1$ is $n^2$, so we can cancel out those terms. We're now
	left with $2n+1=2n+1$, which is a tautology!
\end{example}

\begin{definition}{Recursion}\label{def:recursion}
	A way to express a definition of an element in terms of itself. There's
	two required steps to recursively define a function:
	\begin{itemize}
		\item \textbf{Basis Step:} Specify the value of the function at 0
		\item \textbf{Recursive Step:} Specify a rule for finding its value at
		      an integer from its smaller values at smaller integers
	\end{itemize}
\end{definition}

\section{Counting}\label{sec:counting}
\subsection{Counting Basics}\label{sec:counting-basics}
\marginnote{Think of nested \texttt{for} loops}
\begin{theorem}{The Product Rule}\label{th:product-rule}
	Suppose a procedure can be broken down into a sequence of multiple tasks.
	If there are $n_1$ ways to do the first task, and for each task $k\in n_1$
	there are $n_2$ and continuing on until with the final task having $n_a$
	ways to do that task,  then there are \[n_1\times n_2\times\cdots\times n_a\]
	ways to do that procedure.
\end{theorem}

\begin{example}{Product Rule}\label{ex:product-rule}
	An ice-cream shop has 12 different types of cones, and 32 flavors of ice-cream.
	What is the number of unique single-scoop ice-creams the ice-cream shop can
	serve?
	\br{}
	For each ice-cream cone there are 32 unique flavors of ice-cream that can
	be served with it. This means that there are\[12\times 32=384\] unique
	single-scoop ice-cream combinations that the shop can serve.
\end{example}

\marginnote{Think of sequental \texttt{for} loops}
\begin{theorem}{The Sum Rule}\label{th:sum-rule}
	Suppose a procedure can be broken down into a sequence of multiple tasks.
	If a task can be done in one of $n_1$ ways, or one of $n_2$ ways, or so
	on and so forth until one of $n_a$ ways, then there are
	\[n_1+n_2+\cdots+n_a\] wasy to do the task.
\end{theorem}

\begin{example}{Sum Rule}\label{ex:sum-rule}
	You spot three potential people to steal wallets from. The first person's
	wallet has 4 different types of credit / reward cards, the second person
	has 18 (wow that's a large wallet!), and the third only has one. However,
	you can only take one card in the wallet that you choose. How many possible
	cards do you have to steal?
	\br
	\[4+18+1=23\]
	different cards to rob \emoji{smiling-imp}
\end{example}

\subsection{Pigeonhole Principle}\label{sec:pigeonhole-principle}
\begin{multicols}{2}
	\begin{theorem}{The Pigeonhole Principle}\label{th:pigeonhole}
		If $k\in\mathbb{Z}^+$ and $k+1$ or more objects are placed into $k$
		containers, then there's at least one container containing more than 1
		object.
	\end{theorem}
	\begin{theorem}{Pigeonhole Consequence I}\label{th:pigeonhole-consequence-1}
		A function $f$ from a set of $k+1$ elements to a set of $k$ elements
		will \textit{never be} one-to-one
	\end{theorem}
\end{multicols}

\begin{example}{Pigeonhole and Ice-cream}\label{ex:pigeonhole-icecream-galore}
	For a total count of 23 scoops of ice-cream to be present in 22 cones,
	there must be at least one cone with two ice-cream scoops
\end{example}

\begin{multicols}{2}
	\begin{theorem}{Generalized Pigeonhole Principle}\label{th:generalized-pigenohole}
		If $N$ objects are placed into $k$ containers, then there must be at least
		be one box containing at least $N/k$ objects
		\[\left\lceil\frac{N}{k}\right\rceil\geq m\]
	\end{theorem}
	\begin{example}{Pigeons and Cards}\label{ex:pigenholes-and-cards}
		From a standard deck of 52 playing cards, how many need to be drawn for
		there to be at least 5 cards in the same suite?
		\[\left\lceil\frac{N}{4}\geq5\right\rceil=21\]
	\end{example}
\end{multicols}
\begin{theorem}{Pigeonhole Consequence II}\label{th:pigeonhole-2}
	Every sequence of $n^2+1$ distinct real numbers contains a subsequence
	of length $n+1$ that is either strictly increasing or decreasing
\end{theorem}

\subsection{Permutations and Combinations}\label{sec:permutations-combinations}
\begin{multicols}{2}
	\begin{definition}{Permutation}\label{def:permutation}
		A set of distinct objects in an ordered arragement. An ordered arrangement
		of $n$ elements of a set is called an \textit{$n$-permutation}
	\end{definition}
	\begin{example}{$S=$\{\emoji{apple},\emoji{banana},\emoji{shark}\}}\label{ex:permutation}
		\begin{itemize}
			\item\{\emoji{banana},\emoji{shark},\emoji{apple}\} is a permutation
			\item\{\emoji{shark},\emoji{apple}\} is a 2-permutation
		\end{itemize}
	\end{example}
\end{multicols}
\begin{theorem}{Number of R-Permutations}\label{th:permutations-equation}
	If $n\in\mathbb{Z}^+,r\in\mathbb{Z}$ where $1\leq r\leq n$, then there are
	\[P(n,r)=n\times(n-1)\times(n-1)\ldots\times(n-r+1)\]
	$r$-permutations with a set of $n$ distinct elements
\end{theorem}

\begin{note}{Simplification of the theorem}\label{note:permutations-equation}
	The theorem can be simplified (and more commonly used) like this:
	\[\forall n,r\in\mathbb{Z}.\left(P(n,r)=\frac{n!}{(n-r)!},0\leq r\leq n\right)\]
\end{note}
\begin{multicols}{2}
	\begin{definition}{Combination}\label{def:combination}
		An unordered subset with $r$ amount of elements from a set
	\end{definition}
	\begin{example}{$S=$\{\emoji{shrimp},\emoji{cooking},\emoji{rice}\}}\label{ex:combination}
		\begin{itemize}
			\item \{\emoji{shrimp},\emoji{rice}\} is a combination
			\item \{\emoji{shrimp},\emoji{cooking}\} is a combination as well
		\end{itemize}
	\end{example}
\end{multicols}

\begin{note}{Combinations vs. Permutations}\label{note:combinations-permutations-difference}
	\[S=\{\text{\emoji{cow},\emoji{chicken},\emoji{turkey}}\}\]
	\begin{align*}
		\{\text{\emoji{cow},\emoji{turkey}}\} &  & \{\text{\emoji{turkey},\emoji{cow}}\}
	\end{align*}
	That would be 2 permutations, but \textit{only one} combination
\end{note}

\begin{theorem}{Number of combinations}\label{th:combinations-equation}
	\[\forall n,r\in\mathbb{Z}.\left(C(n,r)=C(n,n-r)=\frac{n!}{r!\left(n-r\right)!},0\leq r\leq n\right)\]
\end{theorem}

\marginnote{factorials make everything big!}
\begin{example}{Gold, Silver, Bronze}\label{ex:permutations}
	How many unique ways are there to select a first, second, and third place
	winners in a contest with 15 contestants?
	\br
	\[P(15,3)=\frac{(15)!}{\left((15)-(3)\right)!}=2730\]
	different ways!
\end{example}

\section{Discrete Probability}\label{sec:discrete-probability}
\marginnote{its range is between 0 and 1, inclusive}
\begin{definition}{Finite Probability}\label{def:finite-probability}
	If $S$ is an non-empty sample space (set of possible outcomes) of equally
	likely outcomes, and $E$ is an event (subset of sample space), the
	probability of $E$ is \[p(E)=\frac{|E|}{|S|}\]
\end{definition}

\begin{example}{Diamond Rings and Seventeen}\label{ex:finite-probability}
	There are two pouches in front of you. In once pouch, there are ten
	diamond rings, but only four of them are ones that your girlfriend likes.
	In another, there are a set of 13 Seventeen photocards---one for each member
	in the group. Only one of them is Minghao. What's the probability of getting
	a diamond ring that your girlfriend likes and a Seventeen photocard?
	\br
	The probability of getting a diamond ring that your girlfriend likes is
	\[p(E_1)=\frac{4}{10}=\frac{2}{5}\]
	While the probability of getting a Minghao photocard is
	\[p(E_2)=\frac{1}{13}\]
	To get the combined probability, remember the
	\hyperref[th:product-rule]{Product Rule}
	\[p(E)=\frac{2}{5}\times\frac{1}{13}=\frac{2}{65}\]
\end{example}

\begin{theorem}{Complementary Probability}\label{th:complementary-probability}
	Let $E$ be an event in a sample space $S$. The probability of the event
	$\bar E=S-E$, the complementary event of $E$ is \[p(\bar E)=1-p(E)\]
\end{theorem}

\begin{theorem}{Union of Two Events}\label{th:union-of-events}
	The union of two events $E_1$ and $E_2$ in a sample space $S$ is
	\[p(E_1\cup E_2)=p(E_1)+p(E_2)-p(E_1\cap E_2)\]
\end{theorem}

\begin{example}{Monty Hall Three-Door Puzzle}\label{ex:three-door-puzzle}
	You're in a game show, and the host invites you to open one of three doors
	that are in front of you. Behind only one of those doors is a scholarship
	that will pay for the rest of your education at UTD. After picking one
	of those doors, the game show host opens a door that a prize isn't behind,
	and asks if you want to switch again. What should you do?
	\br
	The odds that one of the doors has the price is $p(E)=\frac{1}{3}$.
	However, the game show host opening one of the doors without the prize
	in it raises the odds to the other door that you didn't pick to $p(E)=\frac{2}{3}$. This is because the game show host is taking the set of 2 doors that
	you didn't pick and keeping only the best door for you.
	\bl
	Basically, the door you initially picked is just normal and plain, but the
	other one survived the guantlet in a battle against all the other doors and
	won.
\end{example}

\section{Relations}\label{sec:relations}
\begin{definition}{Binary Relation}\label{def:binary-relation}
	Given sets $A,B$, it's a subset of $A\times B$
\end{definition}

\begin{example}{}
	\begin{align*}
		R_1 & = \{(a,b)|a,b\in\mathbb{Z}\land a\leq b\}\subseteq\mathbb{Z}\times\mathbb{Z}        \\
		R_2 & = \{(a,b)|a,b\in\mathbb{Z}\land a<b\}\subseteq\mathbb{Z}\times\mathbb{Z}            \\
		R_3 & = \{(a,b)|a,b\in\mathbb{Z}\land a=b\}\subseteq\mathbb{Z}\times\mathbb{Z}            \\
		R_4 & = \{(a,b)|a,b\in\mathbb{Z}\land a=b+\mathbb{Z}\}\subseteq\mathbb{Z}\times\mathbb{Z} \\
		R_5 & = \{(a,b)|a,b\in\mathbb{Z}\land a|b\}\subseteq\mathbb{Z}\times\mathbb{Z}
	\end{align*}
\end{example}

\begin{multicols}{2}
	{
		\begin{definition}{Reflexivity}\label{def:reflexivity}
			A relation $R$ on a set $A$ is \textit{reflexive} if and only if:
			\[\forall x.\big(x\in A\implies(x,x)\in R\big)\]
		\end{definition}

		\begin{example}{$A=$\{\emoji{apple},\emoji{banana},\emoji{mango},\emoji{lemon}\}}
			This one is symmetric:
			$R=$\{(\emoji{apple},\emoji{banana}),(\emoji{banana},\emoji{apple}),(\emoji{apple},\emoji{apple})\}
			This one isn't, however:
			$R=$\{(\emoji{lemon},\emoji{banana}),(\emoji{banana},\emoji{apple}),(\emoji{mango},\emoji{mango})\}
		\end{example}
	}
	{
		\begin{example}{$A=$\{\emoji{cowboy-hat-face},\emoji{nerd-face}\}}\label{ex:reflexivity}
			This relation is reflexive:\newline
			$R=$\{(\emoji{cowboy-hat-face}, \emoji{cowboy-hat-face}), (\emoji{nerd-face}, \emoji{nerd-face}) (\emoji{nerd-face}, \emoji{cowboy-hat-face})\}
			However, this one isn't:\newline
			$R=$\{(\emoji{cowboy-hat-face}, \emoji{cowboy-hat-face}), (\emoji{cowboy-hat-face}, \emoji{nerd-face}) (\emoji{nerd-face}, \emoji{cowboy-hat-face})\}
		\end{example}

		\begin{definition}{Symmetry}\label{def:symmetry}
			Relation $R$ on a set $A$ is symmetric if and only if
			\[\forall x\forall y.\big((x,y)\in R\implies(y,x)\in R\big)\]
		\end{definition}
	}
\end{multicols}

\begin{definition}{Anti-Symmetry}
	Relation $R$ on a set $A$ is antisymmetric if and only if
	\[\forall x\forall y.\big((x,y)\in R\land(y,x)\in R\implies(y,x)\in x=y)\]
\end{definition}
\begin{example}{}
	\centering
	\begin{tabular}{cl}
		\multirow{2}{*}{$\leq$}                                 & Not symmetric: $3\leq5,5\not\leq3$                   \\
		                                                        & Antisymmetric: $a\leq b,b\leq a\implies a=b$         \\
		\hdashline
		\multirow{2}{*}{$>$}                                    & Not symmetric: $5>3,3\not>5$                         \\
		                                                        & Antisymmetric: $a>b,b>a\implies a=b$                 \\
		\hdashline
		\multirow{2}{*}{$=$}                                    & Symmetric                                            \\
		                                                        & Antisymmetric: $a=b,b=a\implies a=b$                 \\
		\hdashline
		\multirow{2}{*}{$\{(a,b)|a,b\in\mathbb{Z}\land a=-b\}$} & Symmetric: $(a,-a)\in R\implies (-a,a)\in R$         \\
		                                                        & Not antisymmetric: $(a,-a)\land(-a,a)\in R,a\neq -a$ \\
	\end{tabular}
\end{example}

\begin{definition}{Transitive}
	A relation $R$ on set $A$ such that
	\[\forall x\forall y\forall z.\big((x,y)\in R\land(y,z)\in R\implies(x,z)\in R\big)\]
\end{definition}

\begin{example}{}
	\begin{align*}
		A & = \{a,b,c\}                         \\
		R & = \{(a,b),(a,a),(b,b),(b,c),(c,b)\} \\
	\end{align*}
	It's nothing, nada, it's\ldots disappointing
\end{example}

\begin{note}{Equivalence Relation}
	A relation that is reflexive, symmetric, and transitive
\end{note}

\section{Graphs}\label{sec:graphs}
\marginnote{What's a graph without its nodes?}
\begin{definition}{Graph}\label{def:graph}
	Denoted by $G=(V,E)$, consists of a non-empty set of \textit vertices (or nodes) $V$ and a set of edges $E$
\end{definition}
\begin{multicols}{2}
	\begin{definition}{Adjacent Verticies (Undirected)}\label{adjacent-verticies-undirected}
		When the endpoints of an edge $e$ are vertices $a$ and $b$, the verticies
		are considered to be \textit{adjacent} (or \textit{neighbors}) with one
		another.
		\bl
		That edge is called an \textit{incident} edge, \textit{connecting} verticies
		$a$ and $b$
	\end{definition}
	\begin{definition}{Adjacent Verticies (Directed)}\label{def:adjacent-verticies-directed}
		When $(a,b)$ is an edge of a directed graph $G$, $a$ is \textit{adjacent to}
		$b$, and $a$ is \textit{adjacent from} $b$.
		\bl
		Vertex $a$ is called the \textit{initial vertex}, and $b$ is called
		the \textit{terminal} or \textit{end vertex} of $(u,v)$
	\end{definition}
\end{multicols}

\begin{definition}{Neighborhood $\big(N(v)\big)$}\label{def:neighborhood}
	``The set of all neighbors of $u$'' of a graph $G=(V,E)$, denoted by ($N(v)$)
	\bl
	If $A$ is a subset of $V$, $N(A)$ is the set of verticies in $G$ that are
	adjacent to at least one vertex in $A$: \[V=\bigcup_{v\in A}N(v)\]
\end{definition}

\newpage
\begin{multicols}{2}
	\begin{definition}{Degree (Undirected Graph) \big(\texttt{deg}($v$)\big)}\label{def:degree-undirected}
		The number of edges incident with vertex $v$. A loop contributes
		\textit{twice} to the count
	\end{definition}
	\begin{definition}{Degree (Directed Graph) \big(\texttt{deg}$^{\pm}$($v$)\big)}\label{def:degree-directed}
		\texttt{deg}$^-(v)$ number of edges pointing into $v$
		\newline
		\texttt{deg}$^+(v)$ number of edges pointing out from $v$
	\end{definition}
\end{multicols}

\begin{example}{Neighborhoods and Degrees}\label{ex:neighborhoods-and-degrees}
	\centering
	\captionof{figure}{Neighborhoods Example}\label{fig:neighborhoods}
	\begin{graph}
		\node(1) {\emoji{apple}};
		\node[right of=1] (2) {\emoji{banana}};
		\node[below of=1] (3) {\emoji{mango}};
		\node[right of=2] (4) {\emoji{orange}};
		\node[right of=3] (5) {\emoji{coconut}};

		\draw(1) to [bend right] (2);
		\draw(1) to (2);
		\draw(2) to [bend right] (1);
		\draw(1) to (3);
		\draw(1) to (5);
		\draw(2) to (3);
		\draw(2) to (5);
		\draw(2) to (4);
		\draw(4) to (5);
		\draw(4) to [in=270,out=0,looseness=4] (4);
	\end{graph}
	\bl
	\captionof{table}{Table of Neighbors and Degrees}
	\begin{tabular}{ccc}
		\multicolumn{1}{c}{\textbf{Vertex}} &
		\multicolumn{1}{c}{\textbf{Degree}} &
		\multicolumn{1}{c}{\textbf{Neighborhood}}                      \\
		\midrule
		\emoji{mango}                       &
		2                                   &
		\{\emoji{apple},\emoji{banana}\}                               \\
		\emoji{coconut}                     &
		3                                   &
		\{\emoji{apple},\emoji{banana},\emoji{orange}\}                \\
		\emoji{orange}                      &
		4                                   &
		\{\emoji{orange},\emoji{banana},\emoji{coconut}\}              \\
		\emoji{apple}                       &
		5                                   &
		\{\emoji{mango},\emoji{banana},\emoji{coconut}\}               \\
		\emoji{banana}                      &
		6                                   &
		\{\emoji{apple},\emoji{mango},\emoji{coconut},\emoji{orange}\} \\
	\end{tabular}
\end{example}

\begin{note}{Sum of Degrees}\label{note:degree-sum}
	When adding the degrees of all verticies in graph $G=(V,E)$, you'll find
	that each edge contributes \textit{two} to the sum of degrees, since
	each edge is incident with exactly two verticies
\end{note}

\begin{theorem}{Handshake Theorem}\label{th:handshake}
	If $G=(v,E)$ is an undirected graph with $|E|=m$ edges, then
	\[\sum_{v\in V}\texttt{deg($v$)}=\sum_{v\in V_1}\texttt{deg($v$)}+\sum_{v\in V_2}\texttt{deg($v$)}=2\times m\]
	This applies even if multiple edges and loops are present
\end{theorem}

\begin{example}{Handshake}\label{ex:handshake}
	We can see this theorem in effect if we look back at
	\hyperref[fig:neighborhoods]{Figure 1}:
	\begin{align*}
		2\times m = & \sum_{v\in V}\texttt{deg($v$)} \\
		=           & 2 + 3 + 4 + 5 + 6              \\
		2\times m = & 20                             \\
		m =         & 10
	\end{align*}
\end{example}

\begin{multicols}{2}
	\begin{theorem}{Consequences of Handshaking}\label{th:handshake-consequence}
		In an undirected graph, the  sum of degree of nodes with odd degrees is
		\textit{even}.
	\end{theorem}
	\begin{theorem}{Directed Handshake}\label{th:handshake-directed}
		Let $G=(V,E)$ be a directed graph:
		\[\sum_{v\in V}\texttt{deg}^-(v)=\sum_{v\in V}\texttt{deg}^+(v)=|E|\]
	\end{theorem}
\end{multicols}
\newpage
\begin{table}[h!]\label{tab:graph-terminology}
	\centering
	\caption{Graph Terminology}
	\begin{tabular}{llll}
		\multicolumn{1}{c}{\textbf{Type}}      &
		\multicolumn{1}{c}{\textbf{Direction}} &
		\multicolumn{1}{c}{\textbf{Edges}}     &
		\multicolumn{1}{c}{\textbf{Loops}}                                   \\
		\midrule
		Simple Graph                           & Undirected & Singular & No  \\
		Multigraph                             & Undirected & Multiple & No  \\
		Pseudograph                            & Undirected & Multiple & Yes \\
		Simple Directed Graph                  & Directed   & Singular & No  \\
		Directed Multigraph                    & Directed   & Multiple & Yes \\
		Mixed Graph                            & Both       & Multiple & Yes \\
	\end{tabular}
\end{table}

\begin{multicols}{3}
	\begin{minipage}{\linewidth}
		\centering
		\captionof{figure}{Simple}
		\begin{graph}
			\node (1) {\emoji{apple}};
			\node (2) [below of=1] {\emoji{banana}};
			\node (3) [right of=1]{\emoji{cat}};
			\node (4) at (0.85,-0.85) {\emoji{dog}};
			\node (5) [below of=3] {\emoji{elephant}};

			\draw (1) -- (3);
			\draw (3) -- (4);
			\draw (4) -- (2);
			\draw (2) -- (5);
		\end{graph}
	\end{minipage}
	\begin{minipage}{\linewidth}
		\centering
		\captionof{figure}{Multigraph}
		\begin{graph}
			\node (1) {\emoji{apple}};
			\node (2) [below of=1] {\emoji{banana}};
			\node (3) [right of=1]{\emoji{cat}};
			\node (4) at (0.85,-0.85) {\emoji{dog}};
			\node (5) [below of=3] {\emoji{elephant}};

			\draw (1) -- (3);
			\draw (1) -- (4);
			\draw (2) -- (4);
			\draw (4) -- (5);
		\end{graph}
	\end{minipage}
	\begin{minipage}{\linewidth}
		\centering
		\captionof{figure}{Pseudograph}
		\begin{graph}
			\node (1) {\emoji{apple}};
			\node (2) [below of=1] {\emoji{banana}};
			\node (3) [right of=1]{\emoji{cat}};
			\node (4) at (0.85,-0.85) {\emoji{dog}};
			\node (5) [below of=3] {\emoji{elephant}};

			\draw (1) -- (3);
			\draw (1) -- (4);
			\draw (2) -- (4);
			\draw (4) -- (5);
			\draw (5) to [out=0,in=60,looseness=5] (5);
		\end{graph}
	\end{minipage}
	\begin{minipage}{\linewidth}
		\centering
		\captionof{figure}{Simple Directed}
		\begin{graph}
			\node (1) {\emoji{apple}};
			\node (2) [below of=1] {\emoji{banana}};
			\node (3) [right of=1]{\emoji{cat}};
			\node (4) at (0.85,-0.85) {\emoji{dog}};
			\node (5) [below of=3] {\emoji{elephant}};

			\draw[<-] (1) -- (3);
			\draw[->] (1) -- (4);
			\draw[<-] (2) -- (4);
			\draw[->] (2) -- (5);
		\end{graph}
	\end{minipage}
	\begin{minipage}{\linewidth}
		\centering
		\captionof{figure}{Dir. Multigraph}
		\begin{graph}
			\node (1) {\emoji{apple}};
			\node (2) [below of=1] {\emoji{banana}};
			\node (3) [right of=1]{\emoji{cat}};
			\node (4) at (0.85,-0.85) {\emoji{dog}};
			\node (5) [below of=3] {\emoji{elephant}};

			\draw[<->] (1) -- (3);
			\draw[->] (1) -- (4);
			\draw[->] (2) -- (4);
			\draw[->] (4) -- (5);
		\end{graph}
	\end{minipage}
	\begin{minipage}{\linewidth}
		\centering
		\captionof{figure}{Mixed Graph}
		\begin{graph}
			\node (1) {\emoji{apple}};
			\node (2) [below of=1] {\emoji{banana}};
			\node (3) [right of=1]{\emoji{cat}};
			\node (4) at (0.85,-0.85) {\emoji{dog}};
			\node (5) [below of=3] {\emoji{elephant}};

			\draw (1) -- (3);
			\draw[->] (1) -- (4);
			\draw (2) -- (4);
			\draw[->] (4) -- (5);
			\draw[<-] (5) to [out=270,in=330,looseness=5] (5);
		\end{graph}
	\end{minipage}
\end{multicols}

\begin{definition}{Infinite Graphs}\label{def:infinite-graph}
	If $|V|=\infty$, then the graph is referred to as an \textit{infinite} graph,
	otherwise it's called a \textit{finite} graph
\end{definition}
\newpage
\begin{table}[h!]\label{tab:special-graphs}
	\centering
	\caption{Special Graphs}
	\begin{tabular}{ll}
		\multicolumn{1}{c}{\textbf{Graph}} &
		\multicolumn{1}{c}{\textbf{Definition}}                                                                          \\
		\midrule
		Complete Graph                     &
		For $n$ nodes, there exists \textit{exactly one} edge between any
		pair of nodes                                                                                                    \\
		Cyclical Graph                     &
		\parbox{30em}{For $n\geq3$, the graph $C_n$ consists of path: $(v_1,v_2),(v_2,v_2)\ldots,(v_{n-1},v_n),(v_n,1)$} \\
		Bipartite Graph                    &
		\parbox{30em}{A simple graph $G=(V,E)$ such that $V$ can be partitioned into
		two disjoint subsets $V_1$ and $V_2$}                                                                            \\
		Complete Bipartite Graph           &
		\parbox{30em}{A bipartite graph where $(u,v), u\in V_1,v\in V_2$}                                                \\
		Subgraph                           & Let $G=(V,E)$. $H=(A,B)$ is a subgraph of $G$ if
		$A\subseteq V\land B\subseteq E$
	\end{tabular}
\end{table}
\begin{multicols}{2}
	\begin{minipage}{\linewidth}
		\centering
		\captionof{figure}{Complete Graph}\label{fig:complete-graph}
		\begin{graph}
			\node[] (1) {\emoji{apple}};
			\node[] (2) [below of=1] {\emoji{banana}};
			\node[] (3) [right of=1]{\emoji{cat}};
			\node[] (4) [below of=3] {\emoji{elephant}};

			\draw(1) to (2);
			\draw(1) to (3);
			\draw(1) to (4);
			\draw(2) to (3);
			\draw(2) to (4);
			\draw(3) to (4);
		\end{graph}
	\end{minipage}
	\begin{minipage}{\linewidth}
		\centering
		\captionof{figure}{Cycles Graph}\label{fig:cyclical-graph}
		\begin{graph}
			\node[] (1) {\emoji{apple}};
			\node[] (2) [below of=1] {\emoji{banana}};
			\node[] (3) at (1.5,-0.85) {\emoji{cat}};

			\draw(1) to [bend right] (2);
			\draw(2) to [bend right] (3);
			\draw(3) to [bend right] (1);
		\end{graph}
	\end{minipage}
	\begin{minipage}{\linewidth}
		\centering
		\captionof{figure}{Bipartite Graph}\label{fig:bipartite-graph}
		\begin{graph}
			\node[] (1) {\emoji{apple}};
			\node[] (2) [below of=1] {\emoji{banana}};
			\node[] (3) [below of=2] {\emoji{cat}};
			\node[other] (4) [right of=1] {\emoji{orange}};
			\node[other] (5) [below of=4] {\emoji{mango}};
			\node[other](6) [below of=5] {\emoji{frog}};

			\draw(1) to (4);
			\draw(1) to (2);
			\draw(2) to (5);
			\draw(3) to (6);
			\draw(5) to (6);
		\end{graph}
	\end{minipage}
	\begin{minipage}{\linewidth}
		\centering
		\captionof{figure}{Complete Bipartite Graph}\label{fig:complete-bipartite-graph}
		\begin{graph}
			\node[] (1) {\emoji{apple}};
			\node[blank] (2) [right of=1]{};
			\node[] (3) [right of=2] {\emoji{cat}};
			\node[other] (4) [below of=1] {\emoji{orange}};
			\node[other] (5) [right of=4] {\emoji{mango}};
			\node[other](6) [right of=5] {\emoji{frog}};

			\draw(1) to (4);
			\draw(1) to (5);
			\draw(3) to (5);
			\draw(3) to (6);
		\end{graph}
	\end{minipage}
	\begin{minipage}{\linewidth}
		\centering
		\captionof{figure}{Subgraph of \hyperref[fig:complete-graph]{Figure 8}}\label{fig:complete-graph}
		\begin{graph}
			\node[] (1) {\emoji{apple}};
			\node[] (2) [below of=1] {\emoji{banana}};
			\node[] (3) [right of=1]{\emoji{cat}};

			\draw(1) to (2);
			\draw(1) to (3);
			\draw(2) to (3);
		\end{graph}
	\end{minipage}
\end{multicols}
\end{document}


