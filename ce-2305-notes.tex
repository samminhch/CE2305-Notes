\documentclass{MathNotes}

\newenvironment{example}[1]{\begin{BlueBox}{Example}{#1}}{\end{BlueBox}}
\newenvironment{definition}[1]{\begin{RedBox}{Definition}{#1}}{\end{RedBox}}
\newenvironment{note}[1]{\begin{YellowBox}{Note}{#1}}{\end{YellowBox}}
\newenvironment{theorem}[1]{\begin{GrayBox}{Theorem}{#1}}{\end{GrayBox}}
\newenvironment{practice}[1]{\begin{PurpleBox}{Practice}{#1}}{\end{PurpleBox}}

\newcommand{\br}{
	\begin{center}
		\line(1,0){4ex}
	\end{center}
}
\newcommand{\bl}{
	\newline$ $\newline
}
\newcommand{\continued}{
	\mbox{}
	\vfill
	\textbf{Continued on Next Page\ldots}\newpage
}
%----------------------Author Information-----------------
\title{Speed Running Calculus}
\author{Minh Nguyen}

%---------------------Document Begin-----------------------
% \usepackage{layout}
\begin{document}
% \begin{center}
%     \layout*
% \end{center}
\newpage
\maketitle
\pagenumbering{gobble}
\vfill
\tableofcontents
\newpage
\section*{Author's Notes}
This actually isn't the first copy of my discrete math notes\ldots I had an earlier copy that got corrupted, so now I'm just going to constantly upload to GitHub because I'm afraid of them getting wiped again \emoji{sob}
\br{}

These notes are based off of the textbook \textit{Discrete Mathematics and Its Applications} by \textit{Kenneth H. Rosen}, as well as from the lectures of \textit{Serdar Erbatur} from the \textit{Fall 2023} semester.
\br{}
If you have any complaints, or suggestions regarding these notes, please
email me at \newline\href{mailto:minh.nguyen7@utdallas.edu}{mdn220004@utdallas.edu}
\newpage
\pagenumbering{arabic}

\section{Propositional Logic}\label{sec:propositional-logic}
\begin{definition}{Proposition}
	Declarative statements that are either \verb|true| or \verb|false|
\end{definition}

\begin{table}[h!]\label{tab:prop-examples}
	\centering
	\caption{Examples of propositions}
	\begin{tabular}{lc}
		\multicolumn{1}{c}{\textbf{Statement}} & \multicolumn{1}{c}{\textbf{Proposition?}} \\
		\midrule
		$1+1=2$                                & \emoji{check-mark}                        \\
		What class are ya takin?               & \emoji{no-entry}                          \\
		I am happy                             & \emoji{check-mark}
	\end{tabular}
\end{table}

Propositons can be combined to form another proposition with the use of \textit{logical operators}
\begin{note}{}
	Common labels for propositions in discrete mathematics include letters like $P,Q,R,S$\ldots
\end{note}

\begin{table}[h!]\label{tab:logical-operators}
	\centering
	\caption{Logical Operators in Discrete Mathematics}
	\begin{tabular}{clc}
		\multicolumn{1}{c}{\textbf{Symbol}}   &
		\multicolumn{1}{c}{\textbf{Meaning} } &
		\multicolumn{1}{c}{\textbf{Expression}}                                            \\
		\midrule
		$\lnot$                               & negation / not             & $\lnot P$     \\
		$\land$                               & conjunction / and          & $P\land Q$    \\
		$\lor$                                & disjunction / or           & $P\lor Q$     \\
		$\oplus$                              & exclusive disjunction/ xor & $P\oplus Q$   \\
		$\implies$                            & implication / conditional  & $P\implies Q$ \\
		$\iff$                                & biconditional              & $P\iff Q$     \\
	\end{tabular}
\end{table}

\begin{note}{Logical Operator Precedence}\label{order-of-operations}
	Order matters when it comes to evaluating logical operators:
	\begin{enumerate}
		\item negaiton
		\item conjunction
		\item disjunctions
		\item conditionals
	\end{enumerate}
	Where negation is evaluated first, and conditionals last.
\end{note}

\newpage
\begin{multicols}{2}
	\begin{theorem}{More on Conditionals}\label{th:conditionals-extra}
		Conditionals can also be expressed in three other ways:
		\bl
		\textbf{Contrapositive:} $\lnot Q\implies\lnot P$
		\bl
		\textbf{Converse:} $Q\implies P$
		\bl
		\textbf{Inverse:} $\lnot P\implies\lnot Q$
	\end{theorem}
	\begin{example}{}
		\textbf{Conditional:} If \underline{it is raining}, then \underline{I am not going to town}
		\bl
		\textbf{Contrapositive:} If \underline{I go to town}, then \underline{it is not raining}
		\bl
		\textbf{Converse:} If \underline{I do not go to town}, then \underline{it is raining}
		\bl
		\textbf{Inverse:} If \underline{it is not raining}, then \underline{I am going to town}
	\end{example}
\end{multicols}

\begin{definition}{Truth Table}
They are used as a way of seeing all possible values of a proposition
\end{definition}

\begin{table}[h!]\label{tab:truth-table-example}
\centering
\caption{Truth Table of $P\implies Q$}
\begin{tabular}{cc|c}
$P$ & $Q$ & $P\implies Q$\\
\hline
F & F & T \\
F & T & T \\
T & F & F \\
T & T & T \\
\end{tabular}
\end{table}

\subsection{Propositional Equivalencies}
\subsection{Predicates and Quantifiers}
\subsection{Nested Predicates}
\subsection{Inference Rules and Proofs}
\section{Set Theory}
\subsection{Cardinality}
\subsection{Set Operations}
\subsection{Functions}
\subsection{Floor and nCeiling Functions}
\section{Algorithms}
\end{document}
